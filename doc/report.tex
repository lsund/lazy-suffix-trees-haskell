\documentclass[a4paper]{article}

\usepackage[english]{babel}
\usepackage[utf8]{inputenc}
\usepackage{listings}
\usepackage{color}
\usepackage{hyperref}
\usepackage{float}

\title{Master Thesis}

\author{Ludvig Sundstr\"{o}m}
\date{\today}

\begin{document}

\maketitle

\section{General}

Suffix trees expresses the internal structure of a string in a much
deeper level than ordinary preprocessing. They are able to solve the
exact string matching problem with the same runtime bounds as
algorithms such as the Knuth-Morris-Pratt (KMP) algorithm
~\cite{website:knp-wiki} or the Boyer-Moore algorithm (BM)
~\cite{website:bm-wiki}. However, their real value comes from linear
use to being able to solve many string problems more complex than
exact string search. It's hard to find another data structure that
solves that many problems both efficiently and easy as suffix
trees. A typical use case where they excel other string searching
algorithms is when you have one pice of text, fixed over some time,
but are presented multiple pattern strings. The task is now to find
all occurences of each pattern in the text. With suffix trees, you
need $O(n + k)$ for each pattern. In contrast, KMP and BM needs
$O(m + n)$ for each input string presented.

\section{History}

\medskip

\bibliography{report}
\bibliographystyle{plain}

\end{document}
